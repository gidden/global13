Nuclear fuel cycle simulation is a practiced by many different nation states,
academic institutions, and some private companies. The actors in this arena,
however, often are interested in different categories of questions which tend to
drive their simulation design decisions. The first simulators implemented over a
decade ago were designed to answer broad reaching questions regarding the basic
fuel stockpile makeups as a function of time given a variety of top-level fuel
cycle decisions. For instance, one may investigate the relative repository
requirements for different fuel cycles as a function of the aggregate elemental
composition of their mass flows to the repository. Fairly quickly, however, one
realizes that the granularity of the class of questions that can be investigate
is rather coarse. Furthermore, for the majority of simulators, expert input is
required to investigate even slightly different cases (e.g., investigating the
effect of higher burnup fuels). Finally, because each simulator is championed by
a different entity and developed in relative isolation, the actual simulation
mechanics differ across the spectrum of simulators, causing benchmarking efforts
to be cumbersome and making the simulators difficult to validate.

Accordingly, there is strong motivation to reflect on the purpose and scope of
fuel cycle simulation in order to determine how one may investigate the suite of
questions that interested parties have regarding the nuclear fuel cycle. The
proposed work herein provides a methodological description of a generic
simulation engine and interface that is agnostic to the different fuel cycle
options that can be investigate. The interaction amongst facilities is
abstracted away into a mathematical programming approximation of matching of
supply and demand. Facilities themselves can behave in an automated or
user-defined fashion, e.g., modules are currently in development to allow for
on-the-fly isotopic calculations of spent reactor fuel based on run-time
parameters. The facility building decision making is abstracted away into
entities which own a given set of facilities. Finally, the demand for various
facilities' services is abstracted away into entities modeled as regions. The
various layers of abstraction from a simulation design point-of-view allow
interested parties to investigate separate effects on their preferred fuel cycle
metrics. Furthermore, the separation of the simulation engine from the various
entities comprising the fuel cycle allows for easier and more straightforward
testing and benchmarking.

Perhaps one of the strongest motivating factors of a renewed look at design and
collaboration on nuclear fuel cycle simulation is the ability to foster and grow
a simulation community. The proposed simulation framework is being implemented
as \Cyclus \cite{cyclus2012}, a nuclear fuel cycle simulator developed at the
University of Wisconsin - Madison in conjunction with a number of other
collabortors, including the University of Texas at Austin, the University of
Utah, and the University of Idaho. \Cyclus is has an open-source codebase,
allowing other interested developers to use or add to the existing simulation
infrastructure. Critically, the fact that the source code is open allows for
transparent investigation of the inner workings of the simulation. For example,
other's efforts to benchmark their findings agaisnt a \Cyclus simulation are
designed to be relatively easy in comparison to closed-source simulators, given
the availability of the source code, input data, and output database.
