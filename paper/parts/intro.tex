
There is strong motivation for pursuing a decision-based, 
discrete-time, discrete-object fuel cycle simulation infrastructure. 
The fuel cycle simulation community has long been a collection of 
individual actors designing simulation tools for specific 
requirements, the results of which are then difficult or impossible 
to compare/benchmark. Additional difficulties arise when proprietary 
software is used as the foundation of the model framework due to 
licensing restrictions and high costs. Accordingly, we have been 
working through a number of design iterations to arrive at a generic 
framework that supports the nuclear systems simulation community in a 
broad sense, i.e. a ``generic-enough'' tool to allow for system design, 
connection, and analysis as well as comparison with other similarly 
designed systems.

When viewing the nuclear fuel cycle through a facility-connections 
lens, it becomes apparent that the life cycle of material constitutes 
a supply chain. Different supply chain structures are currently 
present in the real fuel cycle, e.g. the rigid top-down structure in France 
as opposed to the more global, multi-tiered supply chains resulting 
from U.S. Section 123 Agreements. In order to model the variety of
possible structures, one must allow for independent decision making. 
Due to the aggregation of requirements, a natural fit is the 
burgeoning field of agent-based, supply-chain network simulation.
