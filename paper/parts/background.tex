
There are a number of agent-based supply chain frameworks and 
implementations available in the literature with varying levels of 
accessibility due to proprietary considerations.\cite{swaminathan_modeling_1998}\cite{julka_agent-based_2002}\cite{van_der_zee_modeling_2005}\cite{chatfield_multi-formalism_2007}
However, the nuclear fuel cycle presents a few unique characteristics not explicitly 
treated in the literature. Perhaps the most difficult consideration 
we have identified is the need to specify target fuel recipes and
match suppliers and consumers based on the requested recipe, i.e. 
there are both quantity and quality constraints placed on a requested 
commodity. An additional difficulty arises with the enforcement of 
regional-boundary constraints (e.g. prohibiting HEU trade between 
regions) and inter-enterprise preferences. We propose to tackle both 
issues via a comprehensive supply/demand matching mechanism.

We propose using a supply/demand matching algorithm that is comprised 
of three main procedures: request-for-bids, preference assignment, 
and resolution. The request-for-bids step signals the producers of 
various commodities of the demand and material specification for 
those commodities. The preference assignment step allows the customers
to analyze each bid in order to assign a preference. The managers of 
these customers (be they at the region or enterprise level) are 
allowed to affect the decision making process at this point in order 
to inform the preferences of the customers covered by their policy 
space. The resolution step takes the as input the bids and 
preferences and outputs the material flows for the given time step.\cite{cyclus2012}
