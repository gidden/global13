\subsection{Fuel Cycle Simulators}

Previous implementations of fuel cycle simulators have varied both in
methodology and distribution platform. The general purpose of all simulators is
to model the flow of material around the fuel cycle in order to determine the
viability of various proposed fuel cycles and their relative
performance \textit{vis-a-vis} a variety of metrics including resource
utilization, costs, and proliferation resistance. However, there are a few key
choices that have been historically made by all simulation developers,
including: what program or language to use, how to determine the flow of
material when there are competing sources or sinks for that material, how to
determine which facilities to build and when to build them, and at what level of
fidelity simulation physics should be modeled (e.g., should material decay or
should it not). One could describe these are the major design choices for
simulation development teams to assess, and in general approaches are taken
which span the gamut from the computationally ``easy'' to the computationally
``complex'' and the spectrum of almost full user-control to more substantial use
of automated decision making.



\subsection{Agent Based Supply Chain Simulation}

There are a number of agent-based supply chain frameworks and implementations
available in the literature with varying levels of accessibility due to
proprietary
considerations \cite{swaminathan_modeling_1998,julka_agent-based_2002,van_der_zee_modeling_2005,chatfield_multi-formalism_2007}.
However, the nuclear fuel cycle presents a few unique characteristics not
explicitly treated in the literature. Perhaps the most difficult consideration
we have identified is the need to specify target fuel recipes and match
suppliers and consumers based on the requested recipe, i.e.  there are both
quantity and quality constraints placed on a requested commodity. An additional
difficulty arises with the enforcement of regional-boundary constraints
(e.g. prohibiting HEU trade between regions) and inter-enterprise
preferences. We propose to tackle both issues via a comprehensive supply/demand
matching mechanism.

We propose using a supply/demand matching algorithm that is comprised 
of three main procedures: request-for-bids, preference assignment, 
and resolution. The request-for-bids step signals the producers of 
various commodities of the demand and material specification for 
those commodities. The preference assignment step allows the customers
to analyze each bid in order to assign a preference. The managers of 
these customers (be they at the region or enterprise level) are 
allowed to affect the decision making process at this point in order 
to inform the preferences of the customers covered by their policy 
space. The resolution step takes the as input the bids and 
preferences and outputs the material flows for the given time step.\cite{cyclus2012}

\subsection{Mathematical Programming}
