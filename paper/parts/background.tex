\subsection{Fuel Cycle Simulators}

Previous implementations of fuel cycle simulators have varied both in
methodology and distribution platform. The general purpose of all simulators is
to model the flow of material among a variety of facilities in order to
determine the viability of various proposed fuel cycles and their relative
performance \textit{vis-a-vis} a variety of metrics including resource
utilization, costs, and proliferation resistance. However, there are a few key
choices that have been historically made by all simulation developers,
including: what program or language to use, how to determine which facilities to
build, when to build them, how to determine the flow of material when there are
competing sources or sinks for that material, whether to model them
individually, and at what level of fidelity physics should be modeled in the
simulation (e.g., should material decay or should it not). One could describe
these are the major design choices for simulation development teams to assess,
and in general approaches are taken which span the gamut from the
computationally ``easy'' to the computationally ``complex'' and the spectrum of
almost full user-control to more substantial use of automated decision making.

The majority of fuel cycle simulation codes have been developed using a system
dynamics platform. CAFCA \cite{guerin_impact_2009}, developed at MIT, was
originally a MATLAB application, but is now written in the commercial software
VENSIM \cite{vensim_2010_ventana}. VISION \cite{jacobson_verifiable_2010},
developed by INL and originally based off of of the DYMOND
code \cite{moisseytsev_dymond_2001}, has its simulation infrastructure written
in the commercial software Powersim \cite{studio_powersim_2003}.
DANESS \cite{van_den_durpel_daness_2009}, originally developed at ANL and now
moved to LISTO bvba in Belgium, is an iThink \cite{richmond_ithink_2004}
application. Simulators that aren't written as commercial system dynamics
applications tend to fall on either side of the spectrum of computational
tools. COSI \cite{boucher_cosi_2005}, for instance, is developed by the French
CEA in Java. Cyclus \cite{cyclus2012}, developed at UW-Madison, is a C++
application which provides an XML-input front end and SQL backend. Other fuel
cycle simulations tend to be focused more on scoping calculations, for which
spreadsheets can be applicable \cite{brudieu_evaluation_2011}. In general, all
simulators (unless otherwise specified above) use Microsoft Excel to manage both
input and output.

After choosing a development platform, the next basic simulation design decision
deals with the modeling of facilities that act as nodes in a fuel-cycle material
flow graph. There are, generally, two strategies that are taken by simulators:
model aggregate fleets of facilities or model individual facilities. The former
approach is taken by both CAFCA and
VISION \cite{busquim_e_silva_system_2008,schweitzer_improved_2008}. Both of
these system dynamics based simulators keep track of general fleet parameters as
a function of time, e.g. the number of facilities under construction, in
operation, or being decommissioned. Furthermore, they model the material flows
into and out of the fleets; however, there is no distinction between individual
facilities -- the fleets themselves are modeled by the system dynamics
equations. COSI, while not a system dynamics code, also models the reactors in
their simulation as a fleet \cite{coquelet-pascal_validation_2011}. DANESS,
though, takes a different tactic, tracking the history of each facility
individually \cite{durpel_daness_2003}. Adding this capability, for instance,
allows DANESS to model individual reactors as unfueled or 'on hold'. 

Fuel cycle simulator developers then must decide how building decisions are to
be made in their models. For example, if both fast reactor and light water
reactor technologies are available to the simulation at some time, how should it
decide which to build, given that there is some demand? This decision is also
necessary for supporting facilities, e.g. separations and fuel fabrication
facilities, if they are modeled explicitly. Most simulators follow one of three
paths: allow user-defined deployment, provide a heuristic that automates
deployment, or combine the two. CAFCA, for instance, guarantees that reactors
can be fueled for their entire lifetime, and thus builds reactors given a
priority ordering, skipping candidates if a look-ahead heuristic determines
there is not enough fuel \cite{busquim_e_silva_system_2008}. In general,
reactors in CAFCA are built to minimize light water reactor spent fuel (i.e.,
fast reactors are built if possible). VISION follows a similar path,
incorporating user-defined reactor preferences before applying an automated
heuristic based on fast reactor fuel
availability \cite{schweitzer_improved_2008}. Neither VISION nor CAFCA allows
reactors to be built without guaranteed fuel availability. DANESS again takes a
slightly different approach. It allows users to define reactor demand and will
instantiate reactors based on a look-ahead heuristic that investigates whether a
certain percentage of the reactor's required fuel is currently available. This
is effectively a 'middle of the road' solution, that guarantees reactor fuel
availability up to some user-defined
percentage \cite{guerin_benchmark_2009}. Accordingly, reactors that aren't
fueled go 'on-hold', as previously discussed. It should be noted that each of
the deployment decisions is based on a ``class'' of reactor (e.g. a PWR), and
that each simulator described above has some limit on the number of ``classes''
of reactors. Importantly, each of these deployment methodologies are
``hard-coded'' into each simulator, i.e., they are the basis of the simulator
framework and can not be changed in general.

The final two classes of simulation design decisions deal with the detail at
which one wishes to model the actual materials in the simulation and at what
level of detail to make material flow decisions based on those materials. Most
simulators generally model physics at a basic level. For example, the decay of
isotopes is available but not used in CAFCA
simulations \cite{guerin_impact_2009}. Material flow decisions are generally
made based on aggregate quantities, e.g. kilograms of plutonium or
transuranics \cite{busquim_e_silva_system_2008}. VISION allows for decay of
isotopics in its simulations, but does not generally make material flow
decisions based on the updated isotopics, instead depending on user-defined
input and output reactor fuel recipes \cite{jacobson_verifiable_2010}. COSI
tends to take the most rigorous approach in this regard. It bases material flow
decisions on equivalence methods, and models in-pile and decay calculations
explicitly using a combination of depletion and transport codes, including
CESAR, APOLLO, and ERANOS \cite{meyer_new_2009}. While COSI models isotopic
transmutation and decay in order to inform their reactor fuel availability
calculations, other simulators generally model isotopic changes to inform output
metrics, such as repository capacity (which is a heavy function of input
material radiotoxicity and decay heat).

\subsection{Agent Based Supply Chain Simulation}

Taking a step back from a review of available fuel cycle simulators, an
investigation of the underlying structure of the supply-demand modeling paradigm
shows that it is, in fact, a subset of the family of supply-chain
models. Further, if one wishes to model individual facilities, each of which can
make independent decisions, one naturally arrives at the concept of agent-based
modeling.There are a number of agent-based supply chain frameworks and
implementations available in the literature with varying levels of accessibility
due to proprietary
considerations \cite{swaminathan_modeling_1998,julka_agent-based_2002,van_der_zee_modeling_2005,chatfield_multi-formalism_2007}.
However, the nuclear fuel cycle presents a few unique characteristics not
explicitly treated in the literature. Perhaps the most difficult consideration
we have identified is the need to specify target fuel recipes and match
suppliers and consumers based on the requested recipe, i.e.  there are both
quantity and quality constraints placed on a requested commodity. An additional
difficulty arises with the enforcement of regional-boundary constraints
(e.g. prohibiting HEU trade between regions) and inter-enterprise
preferences. Julka \cite{julka_agent-based_2002} discusses an extension to the
normal supply-demand matching paradigm that allows for initial information to be
provided to consumers before supply and demand are matched. A full discussion of
this adaptation is provided in the methodology section.

\subsection{Mathematical Programming}

The general suite of techniques we propose to use to solve the supply-demand
matching problem at each time step lie in the realm of mathematical
programming. For general linear optimization problems (i.e., those without
special structure), the Simplex Method, originally developed by
George \cite{dantzig_generalized_1955} in the 1950's, is a very well studied
method for solving linear programs. It is computationally efficient, effectively
solving the linear program (LP) by moving from feasible solution to feasible
solution in a hill-climbing fashion, guaranteeing a increase in the optimality
condition. An LP formulation of the nuclear fuel cycle is provided in the
methodology section. It is foreseeable that such a formulation may not be adequate
for all simulations because solutions can be provided in a fractional
manner. While this is appropriate for certain fuel cycle services (e.g. using a
fraction of an enrichment facility's SWU capacity), it does not necessarily
accurately model reactor orders. Certainly, heuristics can be applied, but
mathematical programming includes a suite of tools that can be used to solve
problems with integer constraints. The analog to the Simplex Method (i.e., the
most ubiquitous method used in practice) for optimization problems with integer
decision variables is the Branch and Bound Method, first presented by Land and
Doig \cite{land_automatic_1960}. It also has been used prolifically in the
optimization community. A mixed integer-linear program (MILP) formulation of the
nuclear fuel cycle is provided in the methodology section.

%% \subsection{Automation of Fuel Cycle Simulation}

%% There has recently been a movement towards automating facility processes and
%% decision making on time scales appropriate for the quick computation times
%% required by the current state of the art of fuel cycle simulation. As currently
%% practiced, fuel cycle simulation is used to investigate the effect of a variety
%% of parameters on larger fuel cycle metrics; accordingly, at the
%% investigation-level, simulators are invoked many times in parameter sweeps. At
%% present, simulators tend to force users to make many decisions, a prime example
%% of which is the current norm of requiring users to provide output and input fuel
%% isotopics. The lone counter example is COSI, which performs relatively precise
%% depletion calculations for each reactor configuration, using a transport
%% calculation to determine fluence rates and multigroup cross sections, collapsing
%% those into 1-group cross sections, and using those 1-group cross sections in
%% CESAR to determine the depleted isotopic vector \cite{meyer_new_2009}. An
%% automated, hybrid method has been proposed by Scopatz that uses an ``essential
%% physics fluence-based parameterized reactor burnup model''
%% \cite{scopatz_essential_2011}, covering both thermal and fast fluence-based
%% reactors.  Similarly, Huff provides an generic repository model, which allows
%% for repository capacity decsision to be made on-the-fly during a simulation
%% \cite{huff_integrated_2013}. One missing link in the automated decision making
%% of the fuel cycle lies in the separations and fuel fabrication step, i.e., how
%% one should match separated material with requested fuel recipes. This is known
%% colloquially as the ``Winery Problem''. A mathematical programming formulation
%% of this problem is provided in the methodology section, extending Oliver's
%% initial concept \cite{oliver_geniusv2:_2009}.
