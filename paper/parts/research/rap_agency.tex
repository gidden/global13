In the previous sections, we have assumed a fundamental truth in our formulation
of the RAP: reactors requesting nuclear fuel know exactly what isotopes they
want and in what quantities they want them. This notion has merit in general; in
real life, the companies that manage reactors and their fuel do, in fact, know
the exact specifications of fuel they want. However, this knowledge comes from
two sources: the availability of the fuel they can purchase at some cost and a
thorough analysis of the effect of various fuels to their reactor flux profiles
and thus their ability to reach high levels of burnup, or fuel
utilization. Furthermore, simulators to date that allow for isotopic-level
dependence of input and output fuel have relied almost solely on the notion of
recipes, hence it made sense when the original RAP formulation was conceived to
have the objective function minimize the $L_1$ norm of the isotopic
vector. Additionally, this is a viable first step that has yet to be shown in
any sort of proof-of-principle, as this work proposes to do.

The original goal of developing such a separations/fuel fabrication facility
that interacts in a fuel cycle simulation is to provide a semblance of realism
with respect to the global inventory of separated material for reactors that
want fuel made out of such material. Critically, this semblance of realism is
\textit{automated} along the similar lines of Scopatz's work for
reactors~\cite{scopatz_essential_2011} and Huff's work for
repositories~\cite{huff_integrated_2013}. Accordingly, the notion that reactors
must know \textit{a priori} the exact fuel recipe they require at any given time
step is, at best, a large approximation. Something that is perhaps closer to
reality is to allow reactors to determine its most important parameter, whether
or not it is in fact an exact fuel recipe. For example, it is a viable
possibility that a reactor would take any reprocessed fuel type that fits
certain parameters, such as a beginning-of-cycle reactivity and a maximum void
coefficient of reactivity. The former parameter is a measure of fuel
utilization, i.e., how long fuel can stay in a reactor before becoming a net
neutron poison, and the latter is a measure of reactor criticality
safety. Accordingly, such a recipe order would effectively connote a notion of
requiring a minimum fuel utilization while maintaining certain safety
parameters, both of which are functions of the resulting isotopics of the given
solution, but neither of which demand specific levels of isotopics.

The RAP formulation can be extended to allow such a level of abstraction with
respect to its objective function and constraints. Let me note beforehand that
this proposed change, while introducing increased agency into the simulation
(allowing reactors greater autonomy over decision-making processes), it
increases connectivity between specific types of facilities, the proposed
separations facility and reactors in this case. Again, \Cyclus has been designed
to minimize these facility-specific connections, so adding them must be done
with sufficient knowledge as to their usefulness. In any case, the point of this
section is to simply show the extension required to the RAP in order to achieve
this aim mathematically.

Let us assume that we have at our disposal the three parameters used in Equation
\ref{eqs:rap}, namely the $L_1$ norm of the isotopic vector, the total mass, and
$\eta$, the neutron production factor. Further, let us assume that there are
three sets of reactors ordering fuel, corresponding to each parameter. I will
denote them as $R_I$, $R_m$, and $R_\eta$ respectively. 

It is important to reiterate the nature of the approximation problem before
continuing, because we are modifying it from its original purpose. As described
in \S\ref{sec:approx}, the approximation problem is concerned with finding an
approximate solution vector, $\vec{x}$, that gets ``close'' to a correct
solution to the matrix equation, i.e., 

\begin{equation*}
  \mathbf{A} \vec{x} \approx \vec{b}.
\end{equation*}

The RAP then applies a weighting coefficient in the objective function due to
domain knowledge of the problem, whereas the unit vector, $\vec{e}$, is used in
the original approximation formulation. The use of the $L_1$ norm is appropriate
in the original RAP case because the mass of isotopes is being approximated,
which is an extrinsic quantity specific to each isotope in the set of
isotopes. Accordingly, such a powerful technique is not required if one is
approximating an aggregate quantity such as overall mass or the neutron
production factor. The simplified equation for mass approximation is

\begin{equation*}
  \sum_{b \in B} m_{b} x_{b,r} \approx m_{r},
\end{equation*}

and the simplified equation for neutron production factor approximation is

\begin{equation*}
\frac{\sum_{b \in B} \eta_{b}^{+} x_{b,r}}
     {\sum_{b \in B} \eta_{b}^{-} x_{b,r}} 
\approx \eta_{r}.
\end{equation*}

We want the difference between each parameter to be approximately zero, i.e.,

\begin{equation*}
  \sum_{b \in B} m_{b} x_{b,r} - m_{r} \approx 0, \mbox{and}
\end{equation*}

\begin{equation*}
\frac{\sum_{b \in B} \eta_{b}^{+} x_{b,r}}
     {\sum_{b \in B} \eta_{b}^{-} x_{b,r}} 
- \eta_{r} \approx 0.
\end{equation*}

Accordingly, we can denote their differences as $y_{m}$ and $y_{\eta}$ and use
these values in our minimal objective function. Take special note that we want
to minimize each $y$, i.e., get them as close to zero as possible. This allows
us to again linearize the neutronics function.

\begin{equation}\label{eqs:mass_obj}
y_{m} = \left|  \sum_{b \in B} m_{b} x_{b,r} - m_{r} \right|
\end{equation}

\begin{equation}\label{eqs:eta_obj}
y_{\eta} = \left| \sum_{b \in B} \eta_{b}^{+} x_{b,r}
- \eta_{r} \sum_{b \in B} \eta_{b}^{-} x_{b,r} \right|
\end{equation}

With the additions to the objective function in hand, the RAP can then be
adapted to a slightly different formulation. Let us call this new problem the
Reactor Order Approximation Problem (ROAP). Note that it has effectively
sectioned the original RAP into three separate request spaces, applying an
objective function and constraints specific to each. Each request space is
connected by the overarching set of barrels, i.e., material must be picked from
all barrels simultaneously.

%%% 
\begin{subequations}\label{eqs:roap}
  \begin{align}
    %%
    \min_{z} \:\: & 
    z = \sum_{r \in R_I} \vec{c_{r}}^{\top} \cdot \vec{y_{I,r}}
    + \sum_{r \in R_m} y_{m,r} + \sum_{r \in R_\eta} y_{\eta,r}
    & \label{eqs:roap_obj} \\
    %%
    \text{s.t.} \:\: &
    \vec{y_{I,r}} = \left| M \cdot \vec{x_{r}}  - \vec{t_{r}} \right|
    &
    \: \forall \: r \in R_I \label{eqs:roap_iso_y} \\
    %%
    &
    y_{m,r} = \left| \sum_{b \in B} m_{b} x_{b,r} - m_{r} \right| 
    &
    \: \forall \: r \in R_m \label{eqs:roap_mass_y} \\
    %%
    &
    y_{\eta} = \left| \sum_{b \in B} \eta_{b}^{+} x_{b,r}
    - \eta_{r} \sum_{b \in B} \eta_{b}^{-} x_{b,r} \right|
    &
    \: \forall \: r \in R_\eta \label{eqs:roap_eta_y} \\
    %%
    &
    \epsilon_{m} \geq \left| \sum_{b \in B} m_{b} x_{b,r} - m_{r} \right|
    & 
    \forall \: r \in R_m, \: \forall \: r \in R_{\eta} \label{eqs:roap_mass} \\
    %%
    &
    \epsilon_{\eta} \sum_{b \in B} \eta_{b}^{-} x_{b,r} \geq 
    & \nonumber \\ 
    & 
    \:\:\:\:\:\:\:\:\:\:\:  % u-g-l-y you ain't got no alibi, you can't format =(
    \left| \sum_{b \in B} \eta_{b}^{+} x_{b,r} - 
    \eta_{r} \sum_{b \in B} \eta_{b}^{-} x_{b,r} \right|
    &
    \forall \: r \in R_{I}, \: \forall \: r \in R_{m} \label{eqs:roap_eta} \\
    %%
    &
    \sum_{r \in R_{I}} x_{b,r} +
    \sum_{r \in R_{m}} x_{b,r} + 
    \sum_{r \in R_{\eta}} x_{b,r} \leq 1
    & 
    \forall \: b \in B \label{eqs:roap_conserv} \\
    %%
    &
    x_{b,r} \in \left[ 0, 1 \right]
    & 
    \forall \: b \in B, 
    \forall \: r \in R  \label{eqs:roap_x}
    %%
  \end{align}
\end{subequations}
%%% 

where $R$ is simply the superset of the partitioned requests

\begin{equation*}
  R = R_{I} \cup R_{m} \cup R_{\eta}.
\end{equation*}
