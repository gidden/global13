Deciding how a simulation is structured from an interactions standpoint is a
delicate balance of known necessity and perceived future needs. There are basic
decisions to make: do you want a system with discrete material transfers or
continuous material flows? Discrete transfers more closely match reality and may
provide insights in that regard, however the require more of their modeling
apparatus due to messaging needs and other structures. More complex decisions
include how one wants to determine connections between facilities. Do we assign
supplier-consumer pairs to facilities? Do we allow them to change? Should the
facility make such a decision? Should that decision be affected by any other
entities? Tellingly, after being exposed to a number of different simulators
through a benchmarking exercise, Guerin comments that the ``operation of a fuel
cycle model is as much art as science'' \cite{guerin_benchmark_2009}. These
simulation-engine decisions comprise the art-related portion of fuel cycle
simulation. The goal of any simulator is to make these decisions in as informed
a way as possible according to domain-level knowledge, taking into account known
and perceived requirements. In general, we attempt to minimize the sheer number
of choices made in this regard, instead relying on well known and well
documented practices of computer scientists and systems engineers.

\Cyclus has an additional goal in that we wish our core simulation
infrastructure to be as flexible as possible. Given a few basic tenets of agent
interaction, other developers should be able to create a new agent to ``plug
in'' to the simulation. Accordingly, we must define a minimal set of behaviors
to sufficiently inform the simulation infrastructure to run the simulation. This
freedom allows us to run the simulation program and attach agents at run time,
effectively separating the simulation engine's functionality from the agents in
the simulation. From an ecosystem point of view, being an open source code and
having such capability allows expansion of the user and developer base into
areas and institutions concerned with security and privacy. Furthermore,
developers could participate both privately and publicly, e.g. adding general
capability to the \Cyclus core that is needed for some functionality without
specifying the internals. This open-source ecosystem further provides incentive
to develop the agent-based simulation architecture. Other developers can
concentrate their efforts on individual agent interaction, effectively
encapsulating developer requirements for learning and interacting with the
various simulation systems. For a more complete discussion on the ecosystem
benefits of the \Cyclus development model, see \cite{huff_open_2011}.

Having decided upon agent-based interactions, one must determine a way to govern
these interactions. We want to minimize agent dependency due to our above
discussion, so using preference-based network flow formulations provide us with
a viable solution technique that provides a consistent interface. The remainder
of this section describes how that market resolution interface is informed by
the agents. Basic agent simulation interaction, such as entering and leaving the
simulation are also described.

\subsubsection{Supply/Demand Parameters}

The resolution of supply and demand at any given time step is the result of the
mathematical program techniques discussed in the subsequent sections; however,
there are simulation-engine details that must be described in order to set up
that problem. The proposed resolution mechanism occurs in nominally three
steps. The agent interactions include consumers and producers of a set of
commodities and progresses is steps or ``phases''. The terminology of this
``phase space'' is taken from previous supply chain agent-based modeling
work \cite{julka_agent-based_2002}.

The first phase allows consumers of commodities to denote both the quantity of a
commodity they need to consume as well as the target isotopics, or
quality. Because the action is technically telling possible providers what type
of product or service is required by a facility, this phase is termed the
``Request for Bids Phase''. This action is considered a ``posting'' of demand to
the market exchange. It is possible that multiple commodities could meet a
consumer's demand. Accordingly, consumers are allowed to ``over-post'', i.e.,
request a larger quantity of commodities than they can actually consume. The
collection of commodities and quantities that make up a consumer's demand is
termed a ``demand portfolio'', where each demanded commodity may also be
accompanied by a target isotopic composition (termed its ``quality'' in \Cyclus
parlance). A capacity constraint is required to accompany each portfolio, which
facilitates the ability for consumers to ``over-post''. Additionally, if a
portfolio is comprised of more than one commodity, the consumer must
additionally provide a cardinal preference over the set of commodities. At the
end of the posting phase, the market exchange will have a set of demand
portfolios for each consumer.

For example, consider an LWR that can be filled with MOX or UOX. At a given time
period in which it will order fuel, it would post a demand portfolio to the
market exchange comprised of a quantity and quality of MOX and a quantity and
quality of UOX. The demand portfolio must also indicate which fuel type it would
prefer, i.e., the cardinal ordering mentioned above. Another example is that of
an advanced fuel fabrication facility which fabricates fuel partially from
separated material that has already passed through a reactor. Fuel assemblies
produced by the facility are normally comprised of some prescribed quantity and
quality of fissile material and the remaining volume is filled with fertile
material, including depleted uranium from enrichment or reprocessed uranium from
separations. Accordingly, a demand portfolio for that facility would be
comprised of a quantity of depleted uranium, a quantity of reprocessed uranium,
a total capacity, and a preference over the two.

The second phase allows suppliers to ``respond'' to the set of consumption
portfolios. Each portfolio is comprised of requests for some set of
commodities. Accordingly, for each request, suppliers of that commodity denote
production capacities and an isotopic profile of the commodity they can
provide. This isotopic profile is part of a heuristic mechanism to assign more
fine-grained preferences among suppliers and consumers. Suppliers are allowed to
offer the null set of isotopics as their profile, effectively providing no
information. A supplier may have its production be constrained by more than one
parameter. For example, a processing facility may have both a throughput
constraint (i.e., it can only process material at a certain rate) and an
inventory constraint (i.e., it can only hold some total material). Further, the
facility could have a constraint on the quality of material to be processed,
e.g., it may be able to handle a maximum radiotoxicity for any given time step
which is a function of both the quantity of material is processes but also the
isotopic content of that material. The formulation provided in the following
section allows for multiple of such constraints as long as they are linear
functions of the demanded commodity quantity. This phase is termed the
``Response to Request for Bids Phase'', and at its completion the possible
connections between supplier and producer facilities, i.e., the arcs in the
graph of the fuel cycle exchange, have been established with specific capacity
constraints defined not only by the quantity of commodities that will traverse
the arcs but also by the quality of the commodities.

The third phase of the supply-demand matching operation involves setting
preference values for each supplier-consumer arc, termed the ``Preference
Assignment Phase''. Supplier-consumer preferences are used to eventually set arc
costs, which drive the solution of the variation of the multi-commodity
transportation problem. Note that the cost translation technique itself is a
simulation design decision. By this phase, each consumer has already assigned
preferences as a function of commodity amongst demands in its portfolio. The
consumer is now allowed to inform the solver as to preferences amongst the
responses to each request in each portfolio. From a facility point of view, the
delineation between responses will nominally be made as a function of the
quality of the material in the each response. 

Take the reactor example provided above. Given an LWR request for MOX and UOX,
multiple responses may be received to its request for MOX, each with different
isotopic profiles of the MOX that can be provided. The reactor can then assign
preference values over this set of potential MOX providers. Another example
comes in the case of repositories. A repository may have a defined preference of
material to accept based upon its heat load or radiotoxicity, both of which are
functions of the quality, or isotopics, of a material. In certain simulators,
limits on fuel entering a repository are imposed based upon the amount of time
that has elapsed since the fuel has exited a reactor. It is in this phase that
the \Cyclus engine would allow such capability. The time constraint is, in
actuality, a constraint on heat load or radiotoxicity (one must let enough of
the fission products decay). A repository could analyze possible input fuel
isotopics and set the arc preference of any that violate a given rule to 0,
effectively eliminating that arc. Once facilities have completed their
preference assignments, their managers are able to permute them based on
institutional or regional factors, as explained below.

\subsubsection{Facilities}

Facilities in \Cyclus are abstracted to either consumers or suppliers of
commodities, and some may be both. Supplier agents are provided agency by being
able to communicate to the market-resolution mechanism a variety of production
capacity constraints in phase two of the solution setup. Consumer agents are
provided agency by being able to assign preferences among possible suppliers
based on the supplier's quality of product. Because this agency is encapsulated
for each agent, it is possible to define strategies that can be attached or
detached to the agents at run-time. 

\subsubsection{Institutions}

Institutions in \Cyclus manage a set of facilities. They are tasked with the
actual instantiation of specific facilities, for example, it is the institutions
job to manage which facilities are commissioned and decommissioned at the
appropriate times. The goal of including a notion of institutions is to allow an
increased level of detail when investigating regional-specific scenarios. For
example, there exist multi-national enterprises, such as AREVA, that operate
fuel cycle facilities in a variety of countries, or regions. Furthermore, there
are international governmental organizations, such as the IAEA, which plans to
operate fuel cycle facilities that service other facilities in a variety of
regions, e.g., a fuel bank. Accordingly, institutions in \Cyclus are able to
augment the preferences of supplier-consumer pairs that have been established in
order to simulate a mutual preference to trade material within an institution or
based on institution-level relationships. Of course, situations arise in the
real world where an institution has the capability to service its own
facilities, but choose to use an outside provider because of either cost or time
constraints, and such a situation is allowed in this framework as well. It is
not immediately obvious to what degree institutions should be allowed to affect
their managed facilities' preferences. Accordingly, through the course of
implementation and experience modeling the fuel cycle with this framework, best
practices will be determined.

\subsubsection{Regions}

Regions in \Cyclus are concerned with meeting certain requirements, e.g. power
capacity, fuel cycle service capacity, etc. The notion of which facilities will
meet this capacity is abstracted away from the region into the set of
institutions that operate in that region. The amount of knowledge a region has
regarding the types of facilities that can meet these external requirements and
the extent to which that knowledge is used for facility deployment decisions
depends on the region's implementation. For example, in the case of nuclear
power capacity, any region knows that it needs additional reactors to be
built. A ``naive'' implementation leaves the building of those reactors to the
institutions that operate in the region. An ``informed'' implementation can
order which reactors should be built by which institutions. The fundamental
driver for such a design decision is a desire to abstract the management of
facilities away from the decision of which facilities to build. It is important
to note here that this abstraction allows for different deployment algorithms to
be tested and exchanged in the \Cyclus framework without necessitating changes
to the simulation engine, as is the case with other simulators described in
previous sections and is consistent with the types of simulation design
decisions required to maintain both flexibility and reusability.

Regions are further provided agency by their ability to affect preferences
between supplier-consumer facility pairs in the third phase of the market
resolution algorithm. The ability to perturb arc preferences between a given
supplier and a given consumer allows fuel cycle simulation developers to model
relatively complex interactions at a regional level. Examples of such
interactions include the notion of tariffs -- a region may prefer that
facilities that can trade within its borders do so. Further, one could model the
effect of sanctions -- a region may not allow trade between facilities within
its borders and another specified region. Because material quality information
is also provided via the market resolution procedure, constraints can be
applied on such information. For example, a region could scan the set of
possible material leaving its borders via supplier transactions. It could then
reset preferences of transactions that violate some decision rule, such as a
restriction of plutonium-based fuels from exiting its boundaries. These
principles can even exist on a spectrum that is a function of either quality
(e.g., a limit could be placed on \nucl{239}{Pu} content) or region (e.g., a
limit could be placed on plutonium-based fuels being provided to another
specific region).
