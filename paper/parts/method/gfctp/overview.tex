
In formulating the Generic Fuel Cycle Transportation Problem (GFCTP), note that
the ``players'' in the set of commodity ``markets'' are the individual
facilities involved in the simulation, i.e., the reactors, fabrication
facilities, repositories, etc. In strict mathematical programming parlance, the
GFCTP can be described as a Mixed-Integer, Multicommodity Transportation Problem
(MTP) with Side Constraints. Accordingly are a number of departures from the
classical MTP that was described in \S\ref{sec:MCTP}.

To begin, the multicommodity aspect of the problem is not manifest on arc
capacities. Instead, facility demand constraints incorporate a set of
satisfactory commodities. For example, a reactor may be able to accept UOX or
MOX fuel, but has a demand for total fuel. Additionally, supplier facilities may
have a set of constraints on their ability to supply a given commodity and they
may not be able to directly express those constraints with the unit of the
commodity market, i.e., kilograms. Take for example an enrichment facility. Such
a facility has nominally two constraints: SWU capacity and natural uranium
capacity. The former constraint is temporal, i.e., it is a processing
constraint. The latter constraint is an inventory constraint. However, both are
necessary to fully define the problem. Furthermore, let us note that the output
of this facility is kilograms of enriched uranium. Accordingly, the above
capacities must be translated into this output. Finally, realism is introduced
through integer variables. For a number of facilities, especially reactors, it
may not be realistic for a given fuel order to be split amongst a variety of
suppliers. The realm of integer programming techniques allow us to introduced
binary variables to enforce this reality constraint.

It should be noted that the addition of integer variables changes both the
complexity of the formulation and the complexity of the solution technique. Such
a change requires a Mixed Integer-Linear Program (MILP) formulation and solution
via the branch-and-bound method (see \S\ref{sec:bnb}) which solves NP-Hard
combinatorial optimization problems whereas the Linear Program (LP) version
requires the transportation simplex method (see \S\ref{sec:trans-simplex})
which is solvable in polynomial time.  Accordingly, I describe the full
formulation in two parts below: \S\ref{sec:GFCTP-LP} describes the linear
program formulation with side constraints which I will denote GFCTP-LP
and \S\ref{sec:GFCTP-E} describes the MILP formulation with side constraints
which I will denote GFCTP-E (E stands for ``exclusive'', i.e., integer variables
denote an exclusive selection of consumers and/or producers).
