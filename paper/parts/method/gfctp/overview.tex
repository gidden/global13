Once a particular instance of a supply and demand commodity market has been
described for a given time step, supplying facilities must somehow be matched
with consumer facilities. It is possible to do this matching in a naive fashion
using some heuristic, and such an approach may be appropriate for some
simulations depending on the level of detail required. However, we seek a more
flexible solution technique that will scale with the detail needed for each
simulation. The mathematical programming discipline provides a natural fit for
such solution techniques through a standard framework used to describe problem
instances and a standard solution algorithm. Of the special problem structures
enumerated in mathematical programming texts, the well-studied Transportation
Problem provides a good starting point for the solution framework we seek.

The Transportation Problem is a member of the family of minimum-cost network
flow problems. It is comprised of a set of suppliers, $I$, a set of consumers,
$J$, and supplier-consumer costs, $c_{i,j}$. Suppliers and consumers form a
bipartite graph, providing the problem with signifigant underlying
structure. Its formulation is straightforward:

%%% 
\begin{subequations}\label{eqs:TP}
  \begin{align}
    %%
    \min_{z} \:\: & 
    z = \sum_{i \in I}\sum_{j \in J} c_{i,j} x_{i,j} 
    & \label{eqs:TP_obj} \\
    %%
    \text{s.t.} \:\: &
    \sum_{j \in J} x_{i,j}^{h} \leq s_{i} 
    &
    \forall \: i \in I \label{eqs:TP_sup} \\
    %%
    &
    \sum_{i \in I} x_{i,j} \geq d_{j} 
    & 
    \forall \: j \in J \label{eqs:TP_dem} \\
    %%
    &
    x_{i,j} \geq 0
    &
    \forall \: x \in X \label{eqs:TP_x}
    %%
  \end{align}
\end{subequations}
%%% 

where Equation \ref{eqs:TP_sup} provides a supply-side constraint, stating that
the amount of material leaving a supplier, $i$, must not be greater than its
supply, $s_i$, and Equation \ref{eqs:TP_dem} provides a demand-side constraint,
stating that the amount of material entering a consumer, $j$, must be at least
the demanded amount, $d_j$. Equation \ref{eqs:TP_x} simply states that flow
along arcs must be positive.

In formulating the Generic Fuel Cycle Transportation Problem (GFCTP), note that
the ``players'' in the set of commodity ``markets'' are the individual
facilities involved in the simulation, i.e., the reactors, fabrication
facilities, repositories, etc. In strict mathematical programming parlance, the
GFCTP can be described as a Mixed-Integer, Multicommodity Transportation Problem
(MTP) with Side Constraints. Accordingly are a number of departures from the
classical MTP that was described in \S\ref{sec:MCTP}.

To begin, the multicommodity aspect of the problem is not manifest on arc
capacities. Instead, facility demand constraints incorporate a set of
satisfactory commodities. For example, a reactor may be able to accept UOX or
MOX fuel, but has a demand for total fuel. Additionally, supplier facilities may
have a set of constraints on their ability to supply a given commodity and they
may not be able to directly express those constraints with the unit of the
commodity market, i.e., kilograms. Take for example an enrichment facility. Such
a facility has nominally two constraints: SWU capacity and natural uranium
capacity. The former constraint is temporal, i.e., it is a processing
constraint. The latter constraint is an inventory constraint. However, both are
necessary to fully define the problem. Furthermore, let us note that the output
of this facility is kilograms of enriched uranium. Accordingly, the above
capacities must be translated into this output. Finally, realism is introduced
through integer variables. For a number of facilities, especially reactors, it
may not be realistic for a given fuel order to be split amongst a variety of
suppliers. The realm of integer programming techniques allow us to introduced
binary variables to enforce this reality constraint.

It should be noted that the addition of integer variables changes both the
complexity of the formulation and the complexity of the solution technique. Such
a change requires a Mixed Integer-Linear Program (MILP) formulation and solution
via the branch-and-bound method (see \S\ref{sec:bnb}) which solves NP-Hard
combinatorial optimization problems whereas the Linear Program (LP) version
requires the transportation simplex method (see \S\ref{sec:trans-simplex})
which is solvable in polynomial time.  Accordingly, I describe the full
formulation in two parts below: \S\ref{sec:GFCTP-LP} describes the linear
program formulation with side constraints which I will denote GFCTP-LP
and \S\ref{sec:GFCTP-E} describes the MILP formulation with side constraints
which I will denote GFCTP-E (E stands for ``exclusive'', i.e., integer variables
denote an exclusive selection of consumers and/or producers).
