Once a particular instance of a supply and demand commodity market has been
described for a given time step, supplying facilities must somehow be matched
with consumer facilities. It is possible to do this matching in a naive fashion
using some heuristic, and such an approach may be appropriate for some
simulations depending on the level of detail required. However, we seek a more
flexible solution technique that will scale with the detail needed for each
simulation. The mathematical programming discipline provides a natural fit for
such solution techniques through a standard framework used to describe problem
instances and a standard solution algorithm. Of the special problem structures
enumerated in mathematical programming texts, the well-studied Transportation
Problem provides a good starting point for the solution framework we seek.

The Transportation Problem is a member of the family of minimum-cost network
flow problems. It is comprised of a set of suppliers, $I$, a set of consumers,
$J$, and supplier-consumer costs, $c_{i,j}$. Suppliers and consumers form a
bipartite graph, providing the problem with significant underlying
structure. Its formulation is straightforward:

%%% 
\begin{subequations}\label{eqs:TP}
  \begin{align}
    %%
    \min_{z} \:\: & 
    z = \sum_{i \in I}\sum_{j \in J} c_{i,j} x_{i,j} 
    & \label{eqs:TP_obj} \\
    %%
    \text{s.t.} \:\: &
    \sum_{j \in J} x_{i,j} \leq s_{i} 
    &
    \forall \: i \in I \label{eqs:TP_sup} \\
    %%
    &
    \sum_{i \in I} x_{i,j} \geq d_{j} 
    & 
    \forall \: j \in J \label{eqs:TP_dem} \\
    %%
    &
    x_{i,j} \geq 0
    &
    \forall \: x \in X \label{eqs:TP_x}
    %%
  \end{align}
\end{subequations}
%%% 

where $x_{i,j}$ is the flow some commodity from $i$ to $j$. 
Equation \ref{eqs:TP_sup} provides a supply-side constraint, stating that
the amount of material leaving a supplier, $i$, must not be greater than its
supply, $s_i$, and Equation \ref{eqs:TP_dem} provides a demand-side constraint,
stating that the amount of material entering a consumer, $j$, must be at least
the demanded amount, $d_j$. Equation \ref{eqs:TP_x} simply states that flow
along arcs must be positive. Finally, the set $X$ is the feasible solution
space.

The Transportation Problem can be extended to include multiple commodities,
resulting in the Multicommodity Transportation Problem (MTP):

%%% 
\begin{subequations}\label{eqs:MTP}
  \begin{align}
    %%
    \min_{z} \:\: & 
    z = \sum_{i \in I}\sum_{j \in J}\sum_{h \in H} c^h_{i,j} x^h_{i,j} 
    & \label{eqs:MTP_obj} \\
    %%
    \text{s.t.} \:\: &
    \sum_{j \in J} x^h_{i,j} \leq s^h_{i} 
    &
    \forall \: i \in I, \forall \: h \in H \label{eqs:MTP_sup} \\
    %%
    &
    \sum_{i \in I} x^h_{i,j} \geq d^h_{j} 
    & 
    \forall \: j \in J, \forall \: h \in H \label{eqs:MTP_dem} \\
    %%
    &
    \sum_{h \in H} x^h_{i,j} \leq r_{i,j}
    & 
    \forall \: i \in I, \:\forall \: j \in J \label{eqs:MTP_arc} \\
    %%
    &
    x^h_{i,j} \geq 0
    &
    \forall \: x \in X \label{eqs:MTP_x}
    %%
  \end{align}
\end{subequations}
%%% 

where, conceptually, the main addition is Equation \ref{eqs:MTP_arc}, which is
termed a ``bundle'' constraint, limiting the total flow of all commodities along
an arc, $(i,j)$, to some value, $r_{i,j}$. The MTP is perturbed by adding
special considerations discussed below to formulate the Generic Fuel Cycle
Transportation Problem (GFCTP).

