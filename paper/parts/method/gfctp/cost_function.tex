In any network flow problem, of which transportation problems are a subset, the
cost of transporting commodities is what drives the solution. Accordingly, an
accurate cost function is necessary to determine an accurate solution. Because
the \Cyclus environment is still a nascent simulation platform, accurate pricing
metrics, and what such metrics even are in terms of a centuries-long fuel cycle
simulation, are generally difficult to come by, with the current standard source
being the Advanced Fuel Cycle Cost Basis
report \cite{shropshire__advanced_2009}. Accordingly, the cost function is
currently a measure of simulation entity preference, rather than a concrete
representation of cost.

The notion of preference extends the work of Oliver's affinity metric
\cite{oliver_geniusv2:_2009}. The preference metric is generally consumer
centric, i.e., consumers have a preference over the possible commodities that
could meet their demand. For example, a reactor may be able to use UOX or MOX
fuel, but may prefer to use MOX fuel. Such a preference differential allows the
projection of real-world cost into the simulation. Additionally, the managers of
a given facility, which in the \Cyclus simulation environment include its
Institution and Region, also exert an influence over its preference. An obvious
example is the concept of affinities given in \cite{oliver_geniusv2:_2009}. In
Oliver's work, an affinity or preference existed between facilities in
``similar'' institutions in order to drive the trading between institutions as a
simple model of international relations. This idea is expanded upon to cover a
facility's other managers and the commodities themselves. Additionally, a
preference can be delineated between the proposed qualities of the same
commodity from different vendors, e.g. if two vendors of MOX fuel
exist. Finally, the notion of a preference is a positive one, and we require a
notion of cost to solve the minimum-cost formulation of the multicommodity
transportation problem with side constraints. Therefore one must utilize a
translation function.

Formally, we define a preference function, $\alpha_{i,j}(h)$, which is a
cardinal preference ordering over a consumer's satisfying commodity set.

\begin{equation}
\alpha_{i,j}(h) \: \forall i \in I \: \forall h \in H_{j} 
\end{equation}

This ordering is a function both of the consumer, $j$, and producer, $i$. The
dependence on producer encapsulates the relationship effects due to managerial
preferences. We then define a cost translation function, $f$, that operates on
the commodity preference function to produce an appropriate cost.

\begin{equation}
f : \alpha_{i,j}(h) \to c_{i,j}^{h}
\end{equation}

A naive implementation, and perhaps all that is necessary for a
proof-of-principle, is to define f as an inversion operator.

\begin{equation}
f(x) = \frac{1}{x}
\end{equation}

The necessity for complexity of this translation function is not immediately
obvious and an analysis will be performed to understand its impact.
