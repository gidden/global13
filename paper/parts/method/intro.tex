This section lays out a plan by which a fully agent-based simulation can be
implemented for a generic fuel cycle. The term ``generic'' implies that
facilities involved are not known \textit{a priori} and, accordingly, facilities
can be coupled together as the designer wishes. For example, a designer has the
choice to model a separations facility and advanced fuel fabrication facility as
separate entities whose connected supply and demand are met by a generic engine,
or to model the two facilities as a single combined and coupled
entity. Additionally, the solution framework for this matching engine must be
agnostic regarding the classes of materials involved. Rather than hard-coding in
constraints and capacities for different material classes, they are added
dynamically based on the entities involved in the solution. Included is a
discussion of the proposal for entity interaction within the generic fuel cycle
simulator framework. The goal of such a discussion is to identify the different
design decisions made by the authors of the various simulators and to
encapsulate the design decisions into the appropriate entities in the \Cyclus
simulation. \Cyclus is designed to provide a minimal framework for agent-based
simulation while maximizing the capability of future developers to adapt
reusable sections of code to implement their own design decisions where
appropriate.

%% only add if we're gonna talk about the RAP

%% Furthermore, there has been previous work in automating the input and output of
%% fuel cycle facilities in order to reduce rigorously voluminous user input of
%% required material and to reduce reliance on closed-source depletion codes that
%% greatly increase simulation times, such as ORIGIN~\cite{bell_origen_1973} and
%% CESAR~\cite{vidal_cesar:_2006}. To this effect, Anthony Scopatz has developed
%% the Bright reactor simulator~\cite{scopatz_essential_2011} that captures the
%% essential physics of reactor depletion calculations. Katy Huff has also added to
%% this area of work by developing a repository simulator that analyzes repository
%% effects due to different combinations of materials in different repository
%% geologies~\cite{huff_integrated_2013}. This work adds a separations facility
%% automation to this family of facilities, allowing for the matching of a set of
%% requested materials with the capability of defining preferences over separated
%% materials based on material properties and facility properties. An example of
%% the latter is the desire to test last-in-first-out (LIFO) and first-in-first-out
%% (FIFO) queues of materials while meeting the constraints of the required
%% material properties.
