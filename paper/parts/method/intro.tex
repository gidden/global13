In this chapter I seek to lay out a plan by which a fully agent-based simulation
can be implemented for a generic fuel cycle. By generic, I mean that the
facilities involved are not known \textit{a priori} and, accordingly,
facilities can be coupled together as the designer wishes. For example, a
designer has the choice to model a separations facility and advanced fuel
fabrication facility as separate entities whose connected supply and demand are
met by a generic engine, or to model the two facilities as a single combined and
coupled entity. Additionally, the solution framework for this matching engine
must be agnostic as to the classes of materials involved. Rather than
hard-coding in constraints and capacities for different material classes, they
are added dynamically based on the entities involved in the solution.

Furthermore, there has been previous work in automating the input and output of
fuel cycle facilities in order to reduce rigorously voluminous user input of
required material and to reduce reliance on closed-source depletion codes that
greatly increase simulation times, such as ORIGIN~\cite{bell_origen_1973} and
CESAR~\cite{vidal_cesar:_2006}. To this effect, Anthony Scopatz has developed
the Bright reactor simulator~\cite{scopatz_essential_2011} that captures the
essential physics of reactor depletion calculations. Katy Huff has also added to
this area of work by developing a repository simulator that analyzes repository
effects due to different combinations of materials in different repository
geologies~\cite{huff_integrated_2013}. This work adds a separations facility
automation to this family of facilities, allowing for the matching of a set of
requested materials with the capability of defining preferences over separated
materials based on material properties and facility properties. An example of
the latter is the desire to test last-in-first-out (LIFO) and first-in-first-out
(FIFO) queues of materials while meeting the constraints of the required
material properties.

Finally, I discuss my proposal for entity interaction within the generic fuel
cycle simulator framework. The goal of this section is to identify the different
design decisions made by the authors of the various simulators in
\S\ref{sec:simulators} and to encapsulate the design decisions into the
appropriate entities in the \Cyclus simulation. Again, the design purpose of
\Cyclus is to provide a minimal framework for agent-based simulation while
maximizing the capability of future developers to adapt reusable sections of
code to implement their own design decisions. The interaction between entities
is interesting on two fronts: defining their interaction behavior, for which
inspiration is taken from previous fuel cycle simulation work (see
\S\ref{sec:simulators}), and implementing the design decisions, for which
``mixins'' will be used. It should be noted that in the object-oriented
community, there are nominally two distinct types of mixins, which I will call
the classical type and updated type. The classical type is discussed slightly in
pp. 390-402 of \cite{stroustrup_c++_2000} as well as Item 43 of
\cite{meyers_effective_2005}. The updated type of mixin makes heavy use of
template metaprogramming and is discussed in \cite{ulrich_mixin-based_2001} and
\cite{smaragdakis_mixin_2002}. Due to the nature of the \Cyclus dynamic-loading
simulation infrastructure and the \Cyclus ecosystem demand for barriers to entry
being as-low-as-reasonably-allowable (ALARA), I chose to use the classical type
of mixin.
