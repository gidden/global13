The previous linear program (LP) formulation of the Generic Fuel Cycle
Transportation Problem fully describes many of the types of transactions that
arise at any given time step. However, it importantly glosses over the critical
case of reactor fuel orders, which comprise a large amount of material orders
within the simulation context. Specifically, it allows reactor fuel orders to be
met by more than one supplier with an arbitrary amount of the order met by each
supplier. Put another way, the LP formulation does not contain the discrete
material information required to model the transaction of fuel assemblies. Such
detail is not necessary in every simulation, but we wish to allow this advanced
modeling for those that do need it. In order to provide this capability of
quantizing orders, binary decision variables must be introduced and integer
programming techniques must be utilized to solve the resulting mixed
integer-linear program. I present the updated formulation below. The key
difference is the inclusion binary variables $y_{i,j}^{h}$, which are 1 if
producer $i$ trades commodity $h$ with consumer $j$ and constants
$\tilde{x}_{j}^{h}$, which denote the quantity of a quantized order. Further a
new set is introduced, $J_{e}$, the set of consumers who require quantized, or
exclusive, orders. The original set of consumers, those who allow partial
orders, I denote $J_{p}$. These two sets constitute the set of all consumers.

\begin{equation}\label{eqs:consumer-union}
  J = J_{p} \cup J_{e}
\end{equation}

The Generic Fuel Cycle Transportation Problem with Exclusive Orders (GFCTP-E)
formulation follows:

\begin{subequations}\label{eqs:GFCTP-E}
  \begin{align}
    %%
    \label{eq:GRCTP-E_obj}
    \min_{z} \:\: 
    & 
    z = \sum_{h \in H}\sum_{i \in I}\sum_{j \in J_{p}}c_{i,j}^{h} x_{i,j}^{h} 
    + \sum_{h \in H}\sum_{i \in I}\sum_{j \in J_{e}}c_{i,j}^{h} y_{i,j}^{h} \tilde{x}_{j}^{h}
    && \\
    %%
    \label{eq:GRCTP-E_sup}
    \text{s.t.} \:\: 
    &
    \sum_{j \in J_{p}}\beta_{i,k}(q_{j}^{h}) x_{i,j}^{h}
    + \sum_{j \in J_{e}}\beta_{i,k}(q_{j}^{h}) y_{i,j}^{h} \tilde{x}_{j}^{h} \leq s_{i,k}^{h} 
    &&
    \forall \: i \in I, \: \forall \: k \in K_{i}^{h}, \forall \: {h \in H} \\
    %%
    \label{eq:GRCTP-E_dem_p}
    &
    \sum_{i \in I}\sum_{h \in H_{j}} x_{i,j}^{h} \geq d_{j}(H_{j}) 
    & 
    \forall \: j \in J_{o} &\\
    %%
    \label{eq:GRCTP-E_dem_e}
    &
    \sum_{i \in I}\sum_{h \in H_{j}} y_{i,j}^{h} \tilde{x}_{j}^{h} \geq d_{j}(H_{j}) 
    &
    \forall \: j \in J_{e}  &\\
    %%
    \label{eq:GRCTP-E_sumy}
    &
    \sum_{h \in H}\sum_{i \in I} y_{i,j}^{h} = 1
    &
    \forall \: j \in J_{e}  &\\
    %%
    \label{eq:GRCTP-E_x}
    &
    x_{i,j}^{h} \geq 0
    &
    \forall \: x \in X  &\\
    %%
    \label{eq:GRCTP-E_y}
    &
    y_{i,j}^{h} \in \{0,1\}
    &
    \forall \: y \in Y &
    %%
  \end{align}
\end{subequations}

The sets and variables involved are described in Tables \ref{tbl:GFCTP-E-sets}
and \ref{tbl:GFCTP-E-vars}. Note that $H_{j}$ is a subset of the commodities:

\begin{equation}
  H_{j} \subseteq H \: \forall \: j \in J_{p}, \forall \: j \in J_{e}
\end{equation}

%%% 
\begin{table} [h!]
\centering
\begin{tabularx}{\textwidth-20pt}{|c|X|} % line wraps second column if too long
\hline
Set         & Description \\
\hline
$H$         & all commodities  \\
$I$         & all producers  \\
$J_{p}$     & all consumers who accept parital orders  \\
$J_{e}$     & all consumers who accept only exclusive orders  \\
$X$         & the feasible set of flows between producers and consumers  \\
$Y$         & the feasible set of exclusive flows between 
            producers and consumers  \\
$K_{i}^{h}$ & the set of constraining capacities for 
            producer $i$ of commodity $h$  \\
$H_{j}$     & the set of satisfying commodities for consumer $j$  \\
\hline
\end{tabularx}
\caption{Sets Appearing in the GFCTP-E Formulation}
\label{tbl:GFCTP-E-sets}
\end{table}
%%% 

%%% 
\begin{table} [h!]
\centering
\begin{tabularx}{\textwidth-20pt}{|c|X|} % line wraps second column if too long
\hline
Variable    & Description \\
\hline
$c_{i,j}^{h}$             & the unit cost of commodity $h$ 
                          for producer $i$ and consumer $j$  \\
$x_{i,j}^{h}$             & a decision variable, the flow of commodity $h$ 
                          for producer $i$ and consumer $j$  \\
$q_{j}^{h}$               & the requested quality of commodity $h$ 
                          by consumer $j$  \\
$y_{i,j}^{h}$             & a binary decision variable that is equal to 1 if 
                          there is flow from producer $i$ to consumer $j$ of 
                          commodity $h$ \\
$\tilde{x}_{j}^{h}$       & the amount of commodity $h$ requested by 
                          consumer $j$ \\
$\beta_{i,k}(q_{j}^{h})$  & a capacity translation function for capacity 
                          constraint $k$ of producer $i$ given $q_{j}^{h}$ \\
$s_{i,k}^{h}$             & a supply capacity of producer $i$ corresponding to 
                          capacity constraint $k$ of commodity $h$ \\
$d_{j}(H_{j})$            & the total demand of consumer $j$ over the set of 
                          satisfying commodities $H_{j}$ \\
\hline
\end{tabularx}
\caption{Variables Appearing in the GFCTP-E Formulation}
\label{tbl:GFCTP-E-vars}
\end{table}
%%%

The examples of the various constraints from the previous section also apply
here. The only difference is the notion of the binary variables, $y_{i,j}^{h}$s,
which denote a sort of on/off switch as to whether a consumer's entire requested
amount of material is met by a supplier or not.

It should be noted that this advanced formulation brings in signifigant
complexity to the resolution method at every time step. As with other mixed
integer-linear programs, the GFCTP-E is NP-Complete and must be solved using the
Branch-and-Bound technique described in \S\ref{sec:bnb}. However, simple
heuristics exist. The most common of them is to solve a relaxed version of the
problem in the form of a linear program, and to round values to form an integer
solution. The exploration of additional heuristics will be performed based on
the outcome of the implementation and analysis of this formulation in
the \Cyclus simulation environment.
