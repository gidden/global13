\textit
{ Simulation of the nuclear fuel cycle is an established field with multiple
  players.  Prior development work has utilized techniques such as system
  dynamics to provide a solution structure for the matching of supply and demand
  in these simulations. In general, however, simulation infrastructure
  development has occured in relatively closed circles, each effort having
  unique considerations as to the cases which are desired to be
  modeled. Accordingly, individual simulators tend to have their design
  decisions driven by specific use cases. Presented in this work is a proposed
  supply and demand matching algorithm that leverages the techniques of the
  well-studied field of mathematical programming. A generic approach is achieved
  by treating facilities as individual entities and actors in the supply-demand
  market which denote preferences amongst commodities. Using such a framework
  allows for varying levels of interaction fidelity, ranging from low-fidelity,
  quick solutions to high-fidelity solutions that model individual transactions
  (e.g. at the fuel-assembly level). The power of the technique is that it allows
  such flexibility while still treating the problem in a generic manner,
  encapsulating simulation engine design decisions in such a way that future
  simulation requirements can be relatively easily added when needed.  }
