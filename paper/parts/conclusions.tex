This paper presented a generic methodology for modeling fuel cycle simulation
entity interactions. The work was motivated by the want to provide a flexible
platform on which one could explore a variety of fuel cycles without requiring
simulation-engine level changes in the underlying code base. Furthermore, this
flexible platform is available as an open source project, allowing any developer
access to the simulation engine while also allowing for cloistered development
at sites with sensitive information. 

Mathematical programming techniques were leveraged to allow such an
encapsulation of simulation engine, abstracting away the notion of hard-coded
connections between facilities. Two strategies were provided, one that runs at
very quick time scales and can be adjusted via simple heuristics to model
individual interactions using a linear programming formulation. A more advance,
and also more computationally time consuming, approach was also provided for
more detailed simulations, utilizing a mixed integer-linear programming
approach.

Future work will concentrate first on implementing the proposed approach in the
\Cyclus code base, building upon the already-existing agent-interaction
simulation infrastructure. Concurrently, the market resolution algorithm will be
benchmarked against other fuel cycle simulation codes in order to provide a
basic level of confidence in it as well as to inform both the \Cyclus
development group and the wider simulation community as to the relative
strengths and weaknesses in a generic approach, rather than modeling specific
fuel cycles explicitly. Additional work on the \Cyclus simulation engine will
also continue, and will involve an incorporation of a graphical user interface
front and back end, effectively removing the need to physically alter XML files
or investigate SQL databases.
